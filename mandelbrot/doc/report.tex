\documentclass{article}
\usepackage[UTF8]{ctex}  % 支持中文的宏包
\usepackage{graphicx}   % 插入图片
\usepackage{amsmath}    % 数学公式
\usepackage{amssymb}  
\usepackage{listings}   % 代码块
\usepackage{xcolor}     % 颜色
\usepackage{algorithm}  % 算法
\usepackage{algpseudocode} % 算法伪代码

\title{Mandelbrot 集可视化项目报告}
\author{宋嘉民}
\date{\today}

% 代码样式设置
\lstset{
    language=C++,
    basicstyle=\ttfamily\small,
    keywordstyle=\color{blue},
    commentstyle=\color{green!50!black},
    stringstyle=\color{red},
    numbers=left,
    numberstyle=\tiny,
    frame=single,
    breaklines=true
}

\begin{document}

\maketitle

\begin{abstract}
本报告详细介绍了 Mandelbrot 集的数学理论和计算方法,以及基于 C++ 实现的 Mandelbrot 集可视化项目。该项目支持多种输出格式、丰富的颜色映射和动态缩放效果。此外,项目还利用 CUDA 进行 GPU 加速计算,显著提高了生成大型高精度分形图像的效率。报告将从数学原理、算法设计、项目架构等方面进行阐述。
\end{abstract}

\section{Mandelbrot 集的数学理论}

\subsection{复数与复平面}

复数是形如 $z = a + bi$ 的数,其中 $a, b$ 是实数,$i$ 是虚数单位,满足 $i^2 = -1$。复数可以在复平面上表示,横坐标代表实部 $a$,纵坐标代表虚部 $b$。复数的模定义为 $|z| = \sqrt{a^2 + b^2}$,表示该点到原点的距离。

\subsection{Mandelbrot 集的定义}

Mandelbrot 集是复平面上的点 $c$ 的集合,满足迭代序列 $z_{n+1} = z_n^2 + c$ 在初始值 $z_0 = 0$ 的条件下保持有界(不发散到无穷大)。

用数学语言表示,Mandelbrot 集 $M$ 定义为:
\begin{equation}
M = \{c \in \mathbb{C} : \lim_{n \to \infty} |z_n| \neq \infty \}
\end{equation}

\subsection{收敛性与边界}

如果对于某个复数 $c$,存在常数 $R$ 使得对所有的 $n$ 都有 $|z_n| \leq R$,则 $c$ 属于 Mandelbrot 集。可以证明,如果某次迭代中 $|z_n| > 2$,则序列一定发散,即该点不属于 Mandelbrot 集。这个重要结论为数值计算提供了有效的判停条件。

\subsection{迭代深度与边界结构}

对于不在 Mandelbrot 集内的点,我们关注它经过多少次迭代后首次满足 $|z_n| > 2$。这个迭代次数称为"逃逸时间",它反映了该点与 Mandelbrot 集的"距离"。逃逸时间越大,点越接近 Mandelbrot 集的边界。

Mandelbrot 集的边界呈现出极其复杂的分形结构,在任意尺度下观察都能发现相似但不完全相同的精细结构。这种无限复杂性是分形几何的典型特征,也是 Mandelbrot 集视觉吸引力的来源。

\section{Mandelbrot 集的计算方法}

\subsection{基本算法}

计算某点是否属于 Mandelbrot 集的基本算法如下:

\begin{algorithm}
\caption{计算点 $c$ 的迭代次数}
\begin{algorithmic}[1]
\Procedure{ComputeIterations}{$c, maxIterations$}
    \State $z \gets 0$
    \State $iterations \gets 0$
    \While{$|z| \leq 2$ AND $iterations < maxIterations$}
        \State $z \gets z^2 + c$
        \State $iterations \gets iterations + 1$
    \EndWhile
    \State \Return $iterations$
\EndProcedure
\end{algorithmic}
\end{algorithm}

在实际计算中,我们无法进行无限次迭代,因此设定一个最大迭代次数作为截止条件。如果达到最大迭代次数仍未发散,我们认为该点属于或非常接近 Mandelbrot 集。

\section{项目结构与功能}

\subsection{系统架构}

项目分为三个主要模块:

\begin{itemize}
    \item \textbf{MandelbrotSet类}:负责核心计算,包括CPU和GPU实现
    \item \textbf{Image类}:处理图像生成和颜色映射
    \item \textbf{主程序}:提供命令行接口和用户交互
\end{itemize}

\subsection{支持的功能}

本项目支持以下功能:

\begin{itemize}
    \item 生成PPM和PNG格式的静态图像
    \item 多种颜色映射方案(HSV平滑渐变和正弦波周期变化)
    \item 通过CUDA加速的GPU计算
    \item 生成展示局部细节的缩放动画
\end{itemize}

\subsection{命令行接口}

系统提供了友好的命令行接口:

\begin{lstlisting}
// 基本使用
./mandelbrot --basic                // 生成基本PPM图像
./mandelbrot --png                  // 生成正弦波颜色的PNG图像
./mandelbrot --png s                // 生成HSV平滑颜色的PNG图像
./mandelbrot --zoom                 // 生成缩放动画
./mandelbrot --cuda --png           // 使用CUDA加速生成PNG图像
\end{lstlisting}

\section{结果展示}

\begin{figure}[htbp]
    \centering
    \includegraphics[width=0.8\textwidth]{./figures/mandelbrot.png}
    \caption{生成的Mandelbrot集图像}
    \label{fig:mandelbrot}
\end{figure}

\subsection{视觉效果}

不同的颜色映射方案产生了不同的视觉效果:

\begin{itemize}
    \item \textbf{基本PPM输出}:使用HSV颜色空间,提供连续的色彩变化
    \item \textbf{正弦波映射}:产生周期性颜色变化,突出不同"层次"的边界
    \item \textbf{HSV平滑映射}:创建彩虹般的平滑渐变,适合展示细节结构
\end{itemize}

\section{结论与展望}

本项目成功实现了Mandelbrot集的高效计算和多样化可视化。通过深入研究Mandelbrot集的数学原理,我们开发了能够生成静态图像和动态缩放动画的系统,并利用GPU并行计算提高了性能。

项目的主要结果包括:

\begin{itemize}
    \item 多种颜色映射技术,提供丰富的视觉效果
    \item CUDA加速的高效计算实现
    \item 支持多种输出格式和缩放动画
\end{itemize}

\end{document}