% 设置文档类为文章类型,使用A4纸张大小
\documentclass[a4paper]{article}

% ====== 导入必要的包 ======
% geometry包用于设置页面布局和边距
\usepackage{geometry}
% 设置页面边距:左3.5cm, 右3.5cm, 上2cm, 下4cm
\geometry{left=3.5cm, right=3.5cm, top=2cm, bottom=4cm}
\usepackage[utf8]{inputenc}
\usepackage{ctex}
\usepackage{amsmath}
\usepackage{amsfonts}
\usepackage{amssymb}
\usepackage{graphicx}
\usepackage{tikz}
\usepackage{enumitem}

% ====== 文档信息设置 ======
\title{中值定理的证明}
\author{宋嘉民}
\date{2025.02.21}

% ====== 文档开始 ======= 
\begin{document}

% 生成标题页
\maketitle

% ====== 主要内容部分 ======
% 第一节:定理及证明
\section{定理及证明}

% 定理陈述
\subsection{中值定理(The Intermediate Value Theorem)}
% 使用数学模式输入定理的形式化表述
设 \( a < b \),且函数 \(f:[a,b] \to \mathbb{R}\) 在 \([a,b]\) 上连续。设 \(y\) 是介于 \(f(a)\) 和 \(f(b)\) 之间的实数,即要么 \(f(a) \leq y \leq f(b)\),要么 \(f(a) \geq y \geq f(b)\),则存在 \(c \in [a,b]\) 使得 \(f(c) = y\)

% 证明部分
\subsection{证明}

% ====== TikZ图形绘制 ======
% 创建一个新的tikz图形环境,设置缩放比例为1.2
\begin{tikzpicture}[scale=1.2]
    % 绘制x轴和y轴
    \draw[->] (-0.5,0) -- (4.5,0) node[right] {$x$};
    \draw[->] (0,-0.5) -- (0,2.5) node[above] {$y$};
    
    % 绘制示例函数曲线(二次函数)
    \draw[blue, thick, smooth, samples=100, domain=0:4] 
        plot (\x,{0.2*(\x)^2 - 0.3*\x + 1});
    
    % 定义关键点坐标
    \coordinate (a) at (1,0.9);
    \coordinate (b) at (3,1.9);
    \coordinate (c) at (2.11,1.252);
    
    % 标记函数上的点
    \fill (a) circle (1.5pt) node[below right] {$f(a)$};
    \fill (b) circle (1.5pt) node[below right] {$f(b)$};
    
    % 在x轴上标记点
    \draw (1,0) node[below] {$a$} -- (1,0.1);
    \draw (3,0) node[below] {$b$} -- (3,0.1);
    \draw (2.11,0) node[below] {$c$} -- (2.11,0.1);
    
    % 绘制中间值y的水平线
    \draw[dashed] (-0.3,1.252) -- (4,1.252) node[right] {$y$};

    % 标记f(c)点
    \fill (c) circle (1.5pt) node[right] {$f(c)$};
    
    % 添加图表标题
    \node[above] at (1.7,2.2) {中值定理证明示意图};
\end{tikzpicture}

% ====== 证明的形式化步骤 ======
% Step 1: 讨论初始条件
我们有两种情况:\(f(a) \leq y \leq f(b)\) 或 \(f(a) \geq y \geq f(b)\)。我们假设前者成立,即 \(f(a) \leq y \leq f(b)\);后者的证明类似,同理易得。

% Step 2: 处理边界情况
如果 \(y = f(a)\) 或 \(y = f(b)\),则结论显然,只需取 \(c = a\) 或 \(c = b\) 即可。因此我们假设 \(f(a) < y < f(b)\)。

% Step 3: 定义关键集合
令集合 \(E\) 定义为:\(E := \{x \in [a,b] : f(x) < y\}\)

% Step 4: 分析集合E的性质
显然 \(E\) 是 \([a,b]\) 的子集,因此有界。又因为 \(f(a) < y\),所以 \(a\) 是 \(E\) 的元素,因此 \(E\) 非空。根据最小上界原理,上确界\(c := \sup(E)\)是有限的。由于 \(E\) 为 \(b\) 所界,我们知道 \(c \leq b\);又因为 \(E\) 包含 \(a\),所以 \(c \geq a\)。因此 \(c \in [a,b]\)。

% Step 5: 证明f(c) = y
设 \(n \geq 1\) 为整数。\(c - \frac{1}{n}\) 小于 \(c = \sup(E)\),因此不可能是 \(E\) 的上界。所以存在一点,记为 \(x_n\),它属于 \(E\) 且大于 \(c - \frac{1}{n}\)。又因为 \(c\) 是 \(E\) 的上界,所以 \(x_n \leq c\)。因此\(c - \frac{1}{n} \leq x_n \leq c\)

根据夹逼准则,我们有 \(\lim_{n \to \infty} x_n = c\)。由于 \(f\) 在 \(c\) 处连续,这意味着 \(\lim_{n \to \infty} f(x_n) = f(c)\)。但由于对每个 \(n\),\(x_n\) 都属于 \(E\),所以对每个 \(n\) 都有 \(f(x_n) < y\)。根据比较原理,我们得到 \(f(c) \leq y\)。因为 \(f(b) > f(c)\),所以 \(c \neq b\)。

因为 \(c \neq b\) 且 \(c \in [a,b]\),所以必有 \(c < b\)。特别地,存在 \(N > 0\) 使得对所有 \(n > N\),都有 \(c + \frac{1}{n} < b\)(因为 \(c + \frac{1}{n}\) 当 \(n \to \infty\) 时收敛于 \(c\))。因为 \(c\) 是 \(E\) 的上确界且 \(c + \frac{1}{n} > c\),所以对所有 \(n > N\),都有 \(c + \frac{1}{n} \notin E\)。因为 \(c + \frac{1}{n} \in [a,b]\),所以对所有 \(n \geq N\),都有 \(f(c + \frac{1}{n}) \geq y\)。但 \(c + \frac{1}{n}\) 收敛于 \(c\),且 \(f\) 在 \(c\) 处连续,因此 \(f(c) \geq y\)。而我们已经知道 \(f(c) \leq y\),所以 \(f(c) = y\),证毕。

% ====== 选择理由部分 ======
\section{选择理由}

% 解释选择该定理的原因
我选择证明中值定理,是因为它不仅是微积分中最基本和重要的定理之一,更是连接函数连续性和函数值域的一座桥梁。这个定理直观上告诉我们,如果一个连续函数的两个函数值分别在某个值的上下两侧,那么在这两点之间一定能找到一点,使得函数在该点的值恰好等于这个值。

% 深入分析选择原因
通过深入研究这个定理的证明过程,能够帮助我更深入地理解连续函数的本质特征,这也是我选择证明这个定理的重要原因。

% 使用带有自定义标记的列表来列举选择理由
\begin{itemize}[label=\tiny{$\bullet$}, itemsep=0pt]
    \item 它体现了连续函数的完备性和中间性质;
    \item 它是证明许多其他重要定理的基础,如罗尔定理、拉格朗日中值定理等,在数值计算、物理建模等领域有着广泛应用
    \item 它让我大一上数学分析期末左右旋转上下摇摆十分痛苦,特此单开一帖,记录一下
\end{itemize}

% ====== 参考文献部分 ======
\begin{thebibliography}{99}
    \bibitem{tao} Tao, T. (2006). Analysis I. Hindustan Book Agency, pp. 206-207.
\end{thebibliography}

\end{document}